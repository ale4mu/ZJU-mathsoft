\documentclass[12pt,a4paper]{ctexart}
\pagestyle{plain}
\usepackage{amsmath, amssymb}


\newtheorem{theorem}{定理}[section]  
\newtheorem{definition}{定义}
\newtheorem{property}{性质}
\newtheorem{proposition}{命题}

\title{Mandelbrot 集及其算法实现}
\author{张子乐 \\ 统计学2301}
\date{\today}


\begin{document}

\maketitle

\begin{abstract}
  本文介绍了 Mandelbrot 集合的数学定义和性质,以及在计算机上实现的算法. 
  通过详细的描述和示例代码,展示了如何利用编程语言生成 Mandelbrot 集合的图像. 
\end{abstract}

\section{引言}
Mandelbrot 集是一种经典的分形结构,  由数学家 Benoît B. Mandelbrot 在 20 世纪 70 年代提出.1980年,他首次可视化了这个集合.
\cite{Mandelbrot2017structural}
Mandelbrot集是分形几何中最著名的例子之一,以其无限复杂和自相似的边界结构而闻名.
本文将首先介绍 Mandelbrot 集合的数学定义和性质,然后探讨其实现算法. 

\section{Mandelbrot集的数学原理}

\begin{definition}
    记$f_c(z)=z^2+c$, 其中$c\in\mathbb{C}$. 记$f_c^{\circ(0)}(z)=f_c(z)$, 
    迭代地定义$f^{\circ(k)}_c(z)=f_c(f^{\circ(k-1)}_c(z))\;(k\geq 1)$. 称集合
    \begin{equation*}
        \mathcal{M}=\{c \in \mathbb{C} : \exists r > 0, | f_c^{\circ(n)}(0) | \leq r, \, \forall n \in \mathbb{N}\}
        \cite{Mandelbrot2017structural}
    \end{equation*}
    为\textbf{Mandelbrot集}.
\end{definition}

\begin{property}
    \label{prop::bounded1}
    Mandelbrot集是有界的.
\end{property}
证明见参考文献\cite{Mandelbrot2017structural}.在证明性质\ref{prop::bounded1}的过程中,我们会发现如下性质.

\begin{property}
    \label{prop::bounded2}
    设$c\in\mathbb{C}$, 若$c\in\mathcal{M}$, 
    则$|f_c^{\circ(n)}(0)|\leq 2\;(\forall n\in\mathbb{N})$.
\end{property}
这个定理是Mandelbrot集生成算法的基础. 

\begin{theorem} \label{theorem:alpha}
    对于$\alpha \in \mathbb{Q}$.
    \begin{enumerate}
    \item 当 $|\alpha| < 2 $ 时,存在有限多个完全实代数参数 $c$ 使得 $\alpha$ 是 $f_c$-预周期的.并且这个集合总是非空的.
    \item 当 $|\alpha| \geq 2$ 时存在无穷多个完全实代数参数 $c$ 使得 $\alpha$ 是 $f_c$-预周期的. 
    \end{enumerate}
\end{theorem}
$f_c$-预周期是复动力系统中的术语,指在$f_c$的迭代下,参数 $\alpha$ 的轨道最终会进入一个周期性的行为.
事实上,使得 $\alpha$ 是$f_c$-预周期的所有完全实代数参数\(c\)(及其共轭)都位于广义Mandelbrot集合的实数部分.\cite{Mandelbrotset2024}

\begin{property}
    \label{prop::symmetric}
    Mandelbrot集关于实轴对称.
\end{property}
证明见参考文献\cite{Mandelbrot2017structural}.

\section{算法实现}
为了绘制 Mandelbrot 集合,  可以采用 ``逃逸时间算法''. 该算法的基本步骤如下:

\begin{enumerate}
    \item 确定复平面上的范围和分辨率.
    \item 对每个复数 $ c $, 从 $ z = 0 $ 开始迭代公式 $ z_{n+1} = z_n^2 + c $. 
    \item 由性质\ref{prop::bounded2}, 如果在最大迭代次数内 $ |z| > 2 $,  
    则认为 $ c $ 不属于 Mandelbrot 集合. 
    \item 如果在迭代达到最大迭代次数时 $ |z| <= 2 $,则近似认为 $ c $ 属于 Mandelbrot 集合.
    \item 属于Mandelbrot集合的点呈现白色. 
    \item 根据逃逸速度为不属于集合的点赋予颜色. 
\end{enumerate}

\section{结论}
Mandelbrot 集合不仅是数学研究的重要对象, 也是科学与艺术的结合体,其图像具有极高的美学价值. 
Mandelbrot 集的生成算法是计算机图形学中的经典案例,展示了如何通过简单的数学规则生成复杂的图像.
\bibliography{mandelbrot.bib}
\bibliographystyle{unsrt}
\end{document}
